\documentclass[onecolumn]{revtex4-2}
\usepackage{amsmath}
\begin{document}
\title{Mathematical details of numpwd}
\author{Christopher K\"orber}
\maketitle


This module computes
\begin{multline}\label{def:pwd}
    O_{(l_o s_o)j_o m_{j_o} (l_i s_i)j_i m_{j_i}}(p_o, p_i, \vec{q})
    \\ =
    \sum\limits_{m_{s_o} m_{s_i}}
    \sum\limits_{m_{l_o} m_{l_i}}
    \left\langle
        l_o m_{l_o}, s_o m_{s_o} \big\vert j_o m_{j_o}
    \right\rangle
    \left\langle
        l_i m_{l_i}, s_i m_{s_i} \big\vert j_i m_{j_i}
    \right\rangle
    \\ \times
    \int d x_o d x_i d \phi_o d \phi_i
    Y_{l_o m_{l_o}}^*(x_o, \phi_o)
    Y_{l_i m_{l_i}}(x_i, \phi_i)
    \\ \times
    \left\langle
        \vec p_o; s_o m_{s_o}
        \big\vert
        \hat O(\vec p_o, \vec p_i, \vec q)
        \big\vert
        \vec p_i; s_i m_{s_i}
    \right\rangle
\end{multline}
in three steps:
\begin{enumerate}
    \item Analytically computing the so called integrated spin decomposed matrix element (ISDME) of the operator
    \item Numerically integrating the product of the so called reduced angular polynomial (RDP) and the ISDME
    \item Summing the integral over all allowed channels
\end{enumerate}

\section{Analytically integrating out one angle}

Propagators usually depend on the difference of incoming and outgoing momenta.
Thus the denominator of operators usually only depending on the difference of azimuthal angles
\begin{equation}
    O(\vec p_o, \vec p_i, \vec q)
    =
    \frac{N(\vec p_o, \vec p_i, \vec q)}{D(p_o, p_i, x_o, x_i, \phi_o - \phi_i, \vec q)}
    \, .
\end{equation}
After substituting
\begin{align}
    \Phi &= \frac{\phi_i + \phi_o}{2}
    \, &
    \phi &= \phi_i - \phi_o
\end{align}
one thus finds
\begin{equation}
    O(\vec p_o, \vec p_i, \vec q)
    =
    \frac{1}{D(p_o, p_i, x_o, x_i, \phi, \vec q)}
    N(p_o, p_i, x_o, x_i, \phi, \Phi, \vec q)
    \, ,
\end{equation}
where it is quite feasible to run all $\Phi$ integrations by hand (no denominator).

Because the spherical harmonics are factorizable in $\phi$
\begin{align}
    Y_{l_o m_{l_o}}^*(x_o, \phi_o)
    Y_{l_i m_{l_i}}(x_i, \phi_i)
    &=
    f_{l_o m_{l_o} l_i m_{l_i}} (x_i, x_o)
    \exp\left(
    - i m_{l_o} \phi_o + i m_{l_i} \phi_i
    \right)\,,
\end{align}
it is possible to entirely factor out the $\Phi$ dependece eq.~\eqref{def:pwd}
\begin{align}
    \phi_i &= \Phi + \frac{\phi}{2} \,, &
    \phi_o &= \Phi - \frac{\phi}{2} &
    \Rightarrow
    - m_{l_o} \phi_o + m_{l_i} \phi_i
    &=
    (m_{l_i} -  m_{l_o}) \Phi + \frac{ m_{l_i} +  m_{l_o}}{2} \phi
\end{align}
defining $m_\lambda =  m_{l_o} -  m_{l_i}$, we thus have
\begin{align}
    &\int_{0}^{2\pi} d \phi_o d \phi_i
    Y_{l_o m_{l_o}}^*(x_o, \phi_o)
    Y_{l_i m_{l_i}}(x_i, \phi_i)
    O_{s_o m_{s_o} s_i m_{s_i}}(\vec p_o, \vec p_i, \vec q)
    \\ =&
    \int_{0}^{2\pi} d \phi
    f_{l_o m_{l_o} l_i m_{l_i}} (x_i, x_o)
    \exp\left(i\frac{ m_{l_i} +  m_{l_o}}{2} \phi\right)
    \\&\qquad\times
    \frac{1}{D(p_o, p_i, x_o, x_i, \phi, \vec q)}
    \int_{0}^{2\pi} d \Phi \exp\left(-i m_\lambda \Phi\right)
    N_{s_o m_{s_o} s_i m_{s_i}}(p_o, p_i, x_o, x_i, \phi, \Phi, \vec q)
    \,.
\end{align}
Note that the integration contours are in principle more difficult and nested, but because of the symmetries of spherical coordinates ($\sin, \cos$ of $\phi_i$ and $\phi_o$ ), this simplifies to independent integrations.

\section{Definitions}

Making the following definitions for the \textbf{reduced angular polynomial}
\begin{equation}\label{def:red-ang}
    A_{(l_o l_i)\lambda m_\lambda}(x_i, x_o, \phi)
    \equiv
    \frac{2 \lambda +1}{2 l_o + 1}
    \sum_{m_{l_i}m_{l_io}}
    \left\langle
        l_i m_{l_i}, \lambda m_{\lambda} \big\vert l_o m_{l_o}
    \right\rangle
    f_{l_o m_{l_o} l_i m_{l_i}} (x_i, x_o)
    e^{i (m_{l_i} +  m_{l_o}) \phi/2}
\end{equation}
and the \textbf{integrated spin decomposed matrix element},
\begin{equation}
    \label{def:integrated-spin-pwd}
    O_{s_o m_{s_o} s_i m_{s_i} m_\lambda}(p_o, p_i, x_o, x_i, \phi, \vec q)
    \equiv
    \int_{0}^{2\pi} d \Phi \exp\left(-i m_\lambda \Phi\right)
    \left\langle
        \vec p_o; s_o m_{s_o}
        \big\vert
        \hat O(\vec p_o, \vec p_i, \vec q)
        \big\vert
        \vec p_i; s_i m_{s_i}
    \right\rangle
\end{equation}
We can compute the PWD of the operator as
\begin{multline}\label{def:pwd}
    O_{(l_o s_o)j_o m_{j_o} (l_i s_i)j_i m_{j_i}}(p_o, p_i, \vec{q})
    \\ =
    \sum\limits_{\lambda m_\lambda}
    \sum\limits_{m_{s_o} m_{s_i}}
    \sum\limits_{m_{l_o} m_{l_i}}
    \left\langle
        l_o m_{l_o}, s_o m_{s_o} \big\vert j_o m_{j_o}
    \right\rangle
    \left\langle
        l_i m_{l_i}, s_i m_{s_i} \big\vert j_i m_{j_i}
    \right\rangle
    \left\langle
        l_i m_{l_i}, \lambda m_{\lambda} \big\vert l_o m_{l_o}
    \right\rangle
    \\ \times
    \int d x_o d x_i d \phi
    A_{(l_o l_i)\lambda m_\lambda}(x_i, x_o, \phi)
    O_{s_o m_{s_o} s_i m_{s_i} m_\lambda}(p_o, p_i, x_o, x_i, \phi, \vec q)
\end{multline}

\end{document}
